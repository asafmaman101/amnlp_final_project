\documentclass[11pt,a4paper]{article}
\usepackage[hyperref]{acl2020}
\usepackage{times}
\usepackage{latexsym}
\renewcommand{\UrlFont}{\ttfamily\small}

\usepackage{multirow}
\usepackage{booktabs}
\usepackage{graphicx}
\usepackage{amsmath}
\usepackage{xcolor}
\usepackage{tabularx}
\usepackage{float}
\usepackage{lscape}
\usepackage{rotating}
\usepackage{tabularx}
\usepackage{caption}
%\usepackage{geometry}
%\usepackage{pdfpages}
\graphicspath{ {./figs/} }
\interfootnotelinepenalty=10000
% \captionsetup{font=small}
\usepackage{titlesec}
\titleformat{name=\subsection}[runin]% runin puts it in the same paragraph
       {\normalfont\bfseries}% formatting commands to apply to the whole heading
       {\thesubsection}% the label and number
       {1em}% space between label/number and subsection title
       {}% formatting commands applied just to subsection title
       [.]% punctuation or other commands following subsection title
\titleformat{name=\subsection,numberless}[runin]% runin puts it in the same paragraph
       {\normalfont\bfseries}% formatting commands to apply to the whole heading
       {}% the label and number
       {}% space between label/number and subsection title
       {}% formatting commands applied just to subsection title
       [.]% punctuation or other commands following subsection title
 
% This is not strictly necessary, and may be commented out,
% but it will improve the layout of the manuscript,
% and will typically save some space.
\usepackage{microtype}

\aclfinalcopy % Uncomment this line for the final submission
%\def\aclpaperid{***} %  Enter the acl Paper ID here

%\setlength\titlebox{5cm}
% You can expand the titlebox if you need extra space
% to show all the authors. Please do not make the titlebox
% smaller than 5cm (the original size); we will check this
% in the camera-ready version and ask you to change it back.

\newcommand\BibTeX{B\textsc{ib}\TeX}
\newcommand\todo[1]{\noindent\textcolor{red}{#1}}
\newcommand{\subtitle}[1]{\vspace{5pt}\noindent\textbf{#1}\quad}

\title{Multi-Task Patterns Training for Task Generalization}

\author{Asaf Maman \\
  Blavatnik School of Computer Science, Tel Aviv University \\
  \texttt{asafmaman@mail.tau.ac.il} \\}

\date{}

\begin{document}
\maketitle

\begin{abstract}
In recent years, the appearance of large language models paved the way to calculate predictions via prompts.
One key feature of this mechanism is the ability to encode information about the task inside the prompt.
This can be accomplished by rewriting input sentences as cloze-style phrases and determining predictions based on the probabilities the model assigns.
This method allows high flexibility for multi-tasking, which raises several questions that we aim to answer in this work.
We found evidence that training models for multi-tasking, when done right, can be used without harming each task's performance.
Furthermore, when training on semantically similar tasks, multi-task training can, to some extent, generalize to unseen tasks.
\end{abstract}


\section{Introduction}
When handling classification tasks, the standard method is to fine-tune the model with a task-specific classification head.
Recently, when large pre-trained language models such as GPT \citep{radford2018improving}, BERT \citep{devlin2019bert} and RoBERTa \citep{liu2019roberta} started becoming prevalent, utilizing the knowledge attained from pre-training via prompting has become feasible.
Providing a task description together with examples has proven successful \citep{radford2019language} also for zero-shot and few-shot learning, which requires an understanding of the task.

This approach unveils an alternative way to deal with classification tasks: reformulating the input examples as cloze-style phrases.
Each example is plugged into a fixed sentence structure, called \textit{pattern}, containing \textit{exactly} one mask token.
We chose the prediction according to the scores the model assigns to a subset of tokens, called \textit{verbalizers}, each associated with one of the classes.
The model will assign a given input to the class whose verbalizer received the highest score.
The method is presented visually in Figure~\ref{fig:method}.

This mechanism delivers a few advantages over conventional classification methods.
First, as mentioned before, it provides some sense of task description encoded in the \textit{pattern-verbalizer pair} (PVP).
That way, the model is fed with information about the task and may be able to infer it from the PVP.
It might allow the model to adapt itself to new tasks without a need for a large amount of new-task examples.
This can facilitate switching tasks by only changing the PVP with no need to initialize and train task-specific weights.
Furthermore, deciding according to MLM predictions makes the fine-tuning closer to the pre-training task and leverages the language modeling knowledge attained for the downstream task.

In this work, we investigate how the performance of masked language models (MLMs) is affected when trained on multiple classification tasks.
In addition, we would like to examine whether providing a task description in the form of PVP can help MLMs generalize to \textit{unseen tasks}.
Finally, we will test these questions in a few-shot setting using PET pipeline.

Our experiments show that it is possible to train models for multi-tasking using PVPs,
and in some cases, training a model for multi-tasking improves performance on unseen tasks.
However, the model's ability to infer tasks from PVPs appears to be restricted to tasks that are semantically similar to train tasks.

\section{Method}
\label{sec:method}

\begin{figure}[h]
	\centering
	\includegraphics[width=0.8\linewidth]{mlm_classification}
	\caption{Classification using cloze test~---~reformulating the sample as a new phrase with exactly one mask token.}
	\label{fig:method}
\vspace{-10pt}
\end{figure}

In general, the use of \textit{pattern-verbalizer pair} (PVP) can be formalized as follows: for a given masked language model $M$, vocabulary $V$ and set of labels $\mathcal{L}$, we define a pattern $P(\textbf{\text{x}})$ that takes as input the raw sample $\textbf{x}\in{}V^*$ and outputs a phrase $P(\textbf{x})\in{}V^*$ that contains exactly one mask token.
For each label in $\mathcal{L}$ we define the set of \textit{verbalizers} using the function $v:\mathcal{L}\rightarrow{}V$ that maps each label to a corresponding token.

We will apply the pattern $P(\textbf{x})$ on each input $\textbf{x}$ that we wish to evaluate.
Then we will use $M$ to receive the score it assigns to each token $t\in{}V$ at the masked token position.
We denote this unnormalized score as $M(t\ |\ P(\textbf{x}))$.
Then we'll calculate the unnormalized score for each class label $l\in{}\mathcal{L}$, that is, $M(v(l)\ |\ P(\textbf{x}))$ and denote it by $s_{\text{\textbf{p}}}(l\ |\ \textbf{x})$ as the score our language model gives to class label $l$ providing $\text{\textbf{p}}=(P, v)$ as the PVP.
Finally, we normalize and calculate the probability distribution over the labels using softmax:
\[
	q_\text{\textbf{p}}(l\ |\ \textbf{x}) = \frac{e^{s_{\text{\textbf{p}}}(l\ |\ \textbf{x})}}{\sum_{l^{'}\in{}\mathcal{L}}e^{s_{\text{\textbf{p}}}(l^{'}\ |\ \textbf{x})}}
\]
We then use the cross-entropy between $q_\text{\textbf{p}}(l\ |\ \textbf{x})$ and the true one-hot encoded label to calculate the loss over all examples. 

\section{Experimental Setup}
\label{sec:experiments}

% SINGLE TASK TRAINING ON TRAIN SET SUMMARY TABLE
\begin{table*}[h]
	\centering
	\fontsize{9}{13}
	\selectfont
	\begin{tabular*}{\textwidth}{l@{\extracolsep{\fill}}cccccccccccccccc}
		\toprule
		& \textbf{AG's News}    & \textbf{Yahoo!}    & \textbf{Yelp Full}    & \textbf{Yelp Polarity}    & \textbf{MNLI}    & \textbf{QQP}    & \textbf{All}    & \textbf{Unsup.}    \\ \hline
		\textbf{AG's News}        & \textbf{93.7}   \tiny$\pm$ 0.06      & \textbf{\underline{81.7}}   \tiny$\pm$ 1.78      & 62.6      \tiny$\pm$ 6.10      & 62.7          \tiny$\pm$ 3.62      & 57.6 \tiny$\pm$ 5.06      & 63.9 \tiny$\pm$ 4.35      & \textbf{92.6} \tiny$\pm$ 0.45      & 66.7   \tiny$\pm$ 7.55      \\
		\textbf{Yahoo!}         & \textbf{56.1}    \tiny$\pm$ 1.88      & \textbf{74.8}   \tiny$\pm$ 0.25      & 39.3       \tiny$\pm$ 5.59      & 38.1           \tiny$\pm$ 7.25      & 24.4  \tiny$\pm$ 2.25      & 34.0 \tiny$\pm$ 5.39      & \textbf{74.1} \tiny$\pm$ 0.68      & 43.2   \tiny$\pm$ 7.66      \\
		\textbf{Yelp Full}     & 27.9    \tiny$\pm$ 6.25      & 32.0   \tiny$\pm$ 2.23      & \textbf{67.7}       \tiny$\pm$ 0.53      & \textbf{44.3}           \tiny$\pm$ 3.94      & 34.1  \tiny$\pm$ 5.23      & 35.6 \tiny$\pm$ 13.9      & \textbf{66.6} \tiny$\pm$ 1.58      & 33.5   \tiny$\pm$ 11.6      \\
		\textbf{Yelp Polarity} & 61.9    \tiny$\pm$ 10.4      & 59.6   \tiny$\pm$ 8.50      & \textbf{97.7}       \tiny$\pm$ 0.26      & \textbf{97.7}           \tiny$\pm$ 0.05      & 75.4  \tiny$\pm$ 4.52      & 75.5 \tiny$\pm$ 22.3      & \textbf{97.3} \tiny$\pm$ 0.29      & 67.3   \tiny$\pm$ 15.7      \\
		\textbf{MNLI}          & 36.0    \tiny$\pm$ 1.31      & 35.6   \tiny$\pm$ 0.42      & 39.4       \tiny$\pm$ 9.67      & 42.1           \tiny$\pm$ 8.47      & \textbf{88.4}  \tiny$\pm$ 0.07      & \textbf{51.8} \tiny$\pm$ 1.95      & \textbf{75.6} \tiny$\pm$ 16.3      & 38.4   \tiny$\pm$ 4.34      \\
		\textbf{QQP}           & 44.7    \tiny$\pm$ 14.5      & 43.7   \tiny$\pm$ 12.4      & 45.7       \tiny$\pm$ 15.2      & 41.1           \tiny$\pm$ 6.71      & \textbf{\underline{77.9}}  \tiny$\pm$ 0.74      & \textbf{85.9} \tiny$\pm$ 0.31      & \textbf{68.2} \tiny$\pm$ 27.7      & 43.9   \tiny$\pm$ 9.57      \\ \bottomrule
	\end{tabular*}
	\caption{\textbf{Single-task training performence on all tasks}~---~reported accuracy results are averaged over all three possible patterns with standard deviation reported aside. Different training tasks are arranged in columns and evaluation tasks in rows. The "All" column corresponds to training on all tasks together and "Unsup." column represents the unsupervised baseline. We emphasized scores that show a significant improvement over the baseline and underlined significant improvement for experiments with different train and test tasks.}\label{tab:single-on-train-set-summary}
	\vspace{-4mm}
\end{table*}


We conducted several experiments to test the questions raised above.
In this section we will describe three setups: Single-task, Multi-task and Few-shot Multi-task training.

The datasets used in the following experiments are AG's News, Yahoo!, Yelp, MNLI and QQP.
They were chosen such that our experiments contain variaty of tasks, some tasks share semantic properties.
Further details about the datasets are deferred to Appendix~\ref{apx:datasets}.

For each task, a set of three PVPs was manually chosen.
The PVPs for tasks that appear also in \citet{schick2020exploiting} where chosen to be the same as in their work.
The PVPs for each task are presented in Appendix~\ref{apx:patterns}.

To provide further analysis and possibly more insights about the behavior of the method, we evaluated each model not only on the PVP that it was trained on but on all three possible PVPs of the evaluated task.

Additional implementation details can be found in Appendix~\ref{apx:implementation-details}.
The code can be found in the github repository\footnote{\url{https://github.com/asafmaman101/amnlp_final_project_code}}.

\subsection{Unsupervised Baselines}
In order to see how fine-tuning affects each task's performance, we evaluated the zero-shot performance of the pre-trained model as a baseline.
By doing this, the difference between the reported scores of the fine-tuned model and the baseline demonstrates exacly the effect of the fine-tuning on the train tasks.

\subsection{Single-task Training}
We trained a supervised model for each classification task.
Each model was evaluated on the task it was trained on but also on all the other tasks.
This gives a reference for how training on one singe task benefits when switching to different task in evaluation time.

\subsection{Multi-task Training}
To test whether multi-tasking will enable generalization to unseen tasks, we combined the training sets of all the datasets and trained a model omitting one task each time.
The missing task was the one that we aimed to generalize to.
We trained each sample with a PVP corresponding to the task it belongs to and evaluated all models on each task separatly.

\subsection{Few-shot Multi-task Training}
In \citet{schick2020exploiting}, the authors presented a method for few-shot training called PET.
As an extention for the multi-tasking training, we used PET to train multi-task models with a few examples.
For each task, we used 1000 examples to train an ensemble of models and generate a soft-labeled dataset.
Finally, we trained one classifier on a 20K size soft-labeled data.
This differs from \citet{schick2020exploiting} first by training on multiple tasks but also by using the method described in Section~\ref{sec:method} to train the classifier.

\section{Results}
Results for single-task training are reported in Table~\ref{tab:single-on-train-set-summary},
multi-task training in Table~\ref{tab:multi-on-train-set-summary}
and few-shot multi-task training in Table~\ref{tab:multi-use-logits-summary}.

\subsection{Single-task Training}
First, when training and evaluating on the same task, the performance succeeds in matching the results with a classification head showed in \citet{schick2020exploiting}.
When evaluating single-task models on tasks different from what they were trained on, the results are mixed.
Some evaluations show improvement over the unsupervised baseline (\textit{i.e.} training on MNLI and testing on QQP)
while some reduced performance (for instance, training on MNLI and testing on Yahoo!).

\subtitle{AG's News on Yahoo!}
Improvement in score does not directly imply generalization.
For example, even though AG's News and Yahoo! have different distributions and classes, there is an artifact to consider when comparing them~---~they share two classes in common: Sports and Business.
Therefore, we cannot conclude that the model learned to generalize to unseen tasks.
This result can be seen more as domain adaptation than a task generalization.

\subtitle{MNLI on QQP}
To overcome this issue, we added paraphrase detection to the set of tasks and evaluated it using QQP dataset.
Paraphrase detection has some semantic relationship with NLI task~---~two phrases with entailment connection might in some cases be paraphrases, while natural and contradiction pairs are not.
Although these two tasks share some semantic characteristics, they do not share labels and the task itself is different.

The performance of the model trained on MNLI when evaluating on QQP shows an average score of 77.9\%,
which is 34\% more than the baseline on this task.
In addition, the model that was trained on QQP achieved 85.9\%,
meaning that testing the MNLI model on QQP got only 8\% less than the model that was fine-tuned on this task.
Although this is still far from concluding that the model inferred the task from the description,
such results when training on a different task and without seeing examples from the target distribution is significant, in our opinion.
While this is still far from concluding that the model is inferring tasks from the description, we believe that such results are meaningful considering the models were trained on clearly different tasks without seeing examples from the target distribution.

\subtitle{PVP Variance}
For some tasks, training and evaluating using different PVPs performed very differently.
For example, QQP performed differently with different PVPs on the Yelp Polarity task (can be seen in extended results in Table~\ref{tab:single-on-train-set}), we can see that two of the PVPs scored roughly 88\%, but one PVP got ~49\%.
This emphasizes the importance of PVP choice for performance and generalization.

\subtitle{Supervised Multi-tasking}
The first question raised in this work was how training on all tasks together would affect performance on each task.
The results show that multi-tasking does not hurt much~---~the model achieves slightly the same results on each task as if it was fine-tuned to this task specifically.
On most tasks, the multi-task training reduced the score by roughly 0.5-1.0\% per task.
The only exception is when evaluating on MNLI and QQP, the average score for this pair is low, but one can notice that the variance is high and that the low average score is a result of one low-performing PVP.
Excluding "bad" PVPs, by using a validation set, for instance, can prevent these cases.
The performance of the multi-task model without the "bad" PVPs is also very close to the single-task performance.

\begin{table*}[ht]
	\centering
	\fontsize{9}{13}
	\selectfont
	% \begin{tabularx}{\linewidth}{lcccccc}
	\begin{tabular*}{\textwidth}{l@{\extracolsep{\fill}}cccccc}
		\toprule
		& \textbf{wo/agnews} & \textbf{wo/yahoo} & \textbf{wo/yelp-full} & \textbf{wo/mnli} & \textbf{all} & \textbf{unsup} \\
		\hline
		\textbf{agnews}    & 82.0 \tiny$\pm$0.25        & 93.6 \tiny$\pm$0.23        & 93.4   \tiny$\pm$0.16          & 93.4 \tiny$\pm$0.27       & 92.6 \tiny$\pm$0.45 & 66.7 \tiny$\pm$7.55 \\
		\textbf{yahoo}     & 75.3 \tiny$\pm$0.23        & 51.8 \tiny$\pm$2.50        & 75.1   \tiny$\pm$0.25          & 75.1 \tiny$\pm$0.12       & 74.1 \tiny$\pm$ 0.68  & 43.2 \tiny$\pm$7.66 \\
		\textbf{yelp-full} & 67.9 \tiny$\pm$0.43        & 67.9 \tiny$\pm$0.54        & 30.5   \tiny$\pm$8.51          & 68.1 \tiny$\pm$0.15       & 66.6 \tiny$\pm$ 1.58 & 33.5 \tiny$\pm$11.6 \\
		\textbf{mnli}      & 88.2 \tiny$\pm$0.09        & 88.1 \tiny$\pm$0.37        & 85.2   \tiny$\pm$4.37          & 35.3 \tiny$\pm$0.88      & 75.6 \tiny$\pm$ 16.3 & 38.4 \tiny$\pm$4.34 \\
		\bottomrule
	\end{tabular*}
	\caption{\textbf{Multi-task training performence on all tasks}~---~results of models trained on all tasks exept one. Reported accuracy results are averaged over all three possible patterns with standard deviation reported aside. Omitting different training tasks are arranged in columns (\textit{i.e.} "wo/agnews" means training on all tasks \textbf{except} agnews) and evaluation tasks in rows.}\label{tab:multi-on-train-set-summary}
\end{table*}

\subsection{Multi-task Training}
Testing on the missing task showed different results among the tasks.
 
\subtitle{AG's News and Yahoo!}
When omitting AG's News, the model succeeded in getting 82.0\% accuracy, which is an improvement over the baseline.
This means that training on the other tasks helps to generalize to this task.
However, when comparing this score with the single-task training, we see almost the same result (81.7\%) as a model trained only on Yahoo!.
This implies that most of the knowledge learned in the multi-task training got from the Yahoo! task and not from the remaining tasks.
The similarity artifacts between AG's News and Yahoo! are takes part in this setup as well.

\subtitle{Yelp and MNLI}
On these two tasks, we notice that the scores of the models trained on all other tasks are worse than the baseline model.
This probably stems from the fact that the training tasks are too semantically different, and the model struggles to learn to be agnostic to the task.

\begin{table}[h]
	\fontsize{7}{10}
	\selectfont
	\begin{tabular}{lcccc}
		\toprule
		& \textbf{wo/agnews} & \textbf{wo/yahoo} & \textbf{wo/yelp-full} & \textbf{wo/mnli} \\
		\hline
		\textbf{agnews}    & 78.8 \tiny$\pm$2.53        & 91.9 \tiny$\pm$0.08        & 91.9   \tiny$\pm$0.09          & 91.7 \tiny$\pm$0.10       \\
		\textbf{yahoo}     & 72.3 \tiny$\pm$0.33        & 52.2 \tiny$\pm$1.72        & 72.1   \tiny$\pm$0.13          & 72.2 \tiny$\pm$0.20       \\
		\textbf{yelp-full} & 65.5 \tiny$\pm$0.20        & 65.6 \tiny$\pm$0.25        & 33.8   \tiny$\pm$5.12          & 65.5 \tiny$\pm$0.24       \\
		\textbf{mnli}      & 85.7 \tiny$\pm$0.20        & 85.8 \tiny$\pm$0.11        & 85.7   \tiny$\pm$0.27          & 37.3 \tiny$\pm$3.27      \\
		\bottomrule
	\end{tabular}
	\caption{Multi-task using logits}\label{tab:multi-use-logits-summary}
\end{table}
\subtitle{QQP}
Similar to single-task training, testing multi-task performance on QQP showed significant improvement over the baseline.
Considering the model was not asked to detect paraphrases during training and has not seen paraphrases before evaluation, it is reasonable to claim that the model succeeded in generalizing to this task.\\

To some extent, training a model for multi-tasks benefits with performance on unseen tasks.
However, it still seems that the model's ability to infer the task from PVP is limited to semantically close tasks.
Although the models failed to generalize to any particular task in the experiments, this limitation is somewhat expected.

\subsection{Few-shot Multi-task Training}
Due to limitations of resources, in this setup, we limited experiments to four tasks.
Like in the previous setups, the results here are also mixed.
One can observe similar relationships between the tasks.

As in the previous setup, training on AG's News or Yahoo, the model succeeds in generalizing at some level to the second task.
Moreover, for Yelp Full and MNLI, the model fails to perform without seeing the task at train time.

On the other hand, when comparing the results to those in Table~\ref{tab:multi-on-train-set-summary},
the low performance does not seem to stem from the small amount of examples.
Moreover, in comparison to supervised results, PET performs only slightly worse.
This result is not surprising and mostly follows the results revealed in \citet{schick2020exploiting}, but testing it here shows that it can also get the benefits of multi-tasking using the method suggested in this work.

\section{Conclusion}
In this work we tested how training and evaluating using PVPs affects model's ability to multi-task and generalize to unseen tasks.
The results show that training for multi-tasking is feasible for achieving almost same performance as for singe-task fine-tuning for a specific task.
Although, the ability to generalize to unseen tasks is limited, at least in the experimental setup of the current work.
Scores on different tasks did improve when training on completely different tasks, but this improvement cannot clearly affiliated with the ability to infer the new task and use existing knowledge for solving it. 
However, the presented improvement on unseen tasks can be benificial and in our opinion can be used in some applications of such models in real-world use cases. 

\bibliography{acl2020}
\bibliographystyle{acl_natbib}

% \newpage
\appendix


\section{Appendices}
\label{sec:appendix}

\subsection{Datasets}
\label{apx:datasets}
The following datasets were chosen to examine the questions raised above.
The consideration when picking datasets was to cover different tasks but also to have tasks that share common semantic characteristics. 
Some datasets were also picked from \citet{schick2020exploiting} to control performance on the different tasks.

\vspace{8pt}
\noindent \textbf{AG's News} \quad
A dataset for text categorization which contains 120K train samples and 7.6K test samples.
It contains articles formatted as title and text body.
The task is to classify each article to one of four categories: (1) World, (2) Sports, (3) Business, (4) Tech.

\vspace{8pt}
\noindent \textbf{Yahoo!} \quad
Text categorization dataset. It contains 1.46M pairs of questions and answers.
The train set contains 1.4M samples and test set 60K samples.
The task is to classify each pair to its subject out of 10 possible options: (1) Society (2) Science (3) Health (4) Education (5) Computer (6) Sports (7) Business (8) Entertainment (9) Relationship (10) Politics.

\vspace{8pt}
\noindent \textbf{Yelp} \quad
Yelp Reviews dataset contains 650K train samples and 50K test samples.
There are two version of tasks for this datset.
One, called "Yelp Full" is to classify a review for a 5-star rating.
The other version named "Yelp Polarity" is to predict whether the review sentiment was positive or negative.

\vspace{8pt}
\noindent \textbf{MNLI} \quad
The MNLI dataset contains pairs of sentences to classify between three possible NLI labels: (1) entailment (2) natural (3) contradiction. 
The dataset has 392K train and 9.8K test samples.

\vspace{8pt}
\noindent \textbf{Quora Question Pairs} \quad
QQP contains 363K train samples and 40K test samples.
The task is to predict whether two sentences are paraphrases one of the other.

\subsection{Implementation Details}
\label{apx:implementation-details}
The following hyperparameters were chosen for the experiments in Section~\ref{sec:experiments}.
We chose the Roberta Large as the model to be used across all experiments.
This choice was done in order to be able to compare results with \citet{schick2020exploiting}.
We trained the model with a max sequence size of 256 tokens.
The batch size was chosen so that it will fit into 11GB GPU infrastucture, this led to a batch size of 2. 
In order to simulate larger batch sizes we accumulated gradients for 8 steps to have effective batch size of 32.
Learning rate was chosen to be 1e-5, we trained each model for 3 epochs on the whole train set.
Temperature was tuned to value of 2, weight decay to 0.01, and adam epsilon to 1e-8.

\subsection{Patterns}
\label{apx:patterns}
We chose the following datasets to evaluate performence on.

\vspace{8pt}
\noindent \textbf{AG's News} \quad
Each sample $\textbf{x}$ in the dataset consists of title and text body.
That is $\textbf{x}=(a, b)$ where $a$ and $b$ represents the title and text body respectively.
For the experiments, we used the following three patterns:
\begin{table}[H]
	\renewcommand{\arraystretch}{1.5}
	\begin{tabularx}{\textwidth}{cc}
		 $P_1(x)=$ \_\_\_\_: $a$ $b$ & $P_2(x)=$ \_\_\_\_ News: $a$ $b$ \\
		 $P_3(x)=$ $a$ (\_\_\_\_) $b$ &  \\
	\end{tabularx}
\end{table}
The verbalizers used for each class were World, Sports, Business and Tech corresponding to the dataset categories.

\vspace{8pt}
\noindent \textbf{Yahoo} \quad
Each sample $\textbf{x}$ in the dataset consists of question and answer.
That is $\textbf{x}=(a, b)$ where $a$ and $b$ represents the question and answer respectively.
For the experiments, we used the following three patterns:
\begin{table}[H]
	\renewcommand{\arraystretch}{1.5}
	\begin{tabularx}{\textwidth}{cc}
		$P_1(x)=$ \_\_\_\_: $a$ $b$ & $P_2(x)=$ \_\_\_\_ Question: $a$ $b$ \\
		$P_3(x)=$ $a$ (\_\_\_\_) $b$ &  \\
	\end{tabularx}
\end{table}
The verbalizers used for each class were "Society", "Science", "Health", "Education", "Computer", "Sports", "Business", "Entertainment", "Relationship" and "Politics", corresponding to the dataset categories.

\vspace{8pt}
\noindent \textbf{Yelp} \quad
Each sample $\textbf{x}$ in the dataset consists of a review $a$.
For the experiments, we used the following three patterns:
\begin{table}[H]
	\renewcommand{\arraystretch}{1.5}
	\begin{tabularx}{\textwidth}{l}
		$P_1(x)=$It was \_\_\_\_. $a$ \\
		$P_2(x)=$ $a$. All i n all, it was \_\_\_\_. \\
		$P_3(x)=$ Just \_\_\_\_! $a$  \\
	\end{tabularx}
\end{table}
The Yelp Full and Yelp Polarity were differentiated by the verbalizers used in each task.
For Yelp Full, the 1-5 rating scale was affiliated with "terrible", "bad", "okay", "good", "great" verbalizers.
Yelp Polarity was flatten to "bad" and "good" only.

\vspace{8pt}
\noindent \textbf{MNLI} \quad
Each sample $\textbf{x}$ in the dataset consists of two sentences $a$ and $b$.
The three patterns used for the experiments were:
\begin{table}[H]
	\renewcommand{\arraystretch}{1.5}
	\begin{tabularx}{\textwidth}{cc}
		$P_1(x)=$"$a$"? \_\_\_\_, "$b$" & $P_2(x)=$$a$? \_\_\_\_, $b$ \\
		$P_3(x)=$"$a$"? \_\_\_\_, "$b$" & \\
	\end{tabularx}
\end{table}
And the verbalizers used for each were "Wrong", "Right" and "Maybe" for Contradiction, Entailment and Natural classes respectively.
$P_3$ differs from $P_1$ by its verbalizers, it was used with "No", "Yes" and "Maybe" instead.

\vspace{8pt}
\noindent \textbf{Quora Question Pairs} \quad
Each sample $\textbf{x}$ in the dataset consists of two sentences $a$ and $b$.
The three patterns used for the experiments were:
\begin{table}[H]
	\renewcommand{\arraystretch}{1.5}
	\begin{tabularx}{\textwidth}{cc}
		$P_1(x)=$"$a$"? \_\_\_\_, "$b$" & $P_2(x)=$$a$? \_\_\_\_, $b$ \\ 
		$P_3(x)=$"$a$"? \_\_\_\_, "$b$" & \\
	\end{tabularx}
\end{table}
And the verbalizers used for each were "Wrong" and "Right" for sentences that are not paraphrases and are paraphrases respectively.
$P_3$ differs from $P_1$ by its verbalizers, it was used with "No" and "Yes" instead.

\section{Full Results Report}
\label{apx:full-results-report}
The exact score for each single experiment is describe in and Table~\ref{tab:single-on-train-set} and Table~\ref{tab:multi-use-logits-full}.
\input{single-on-train-set-horizontal}
\clearpage
\thispagestyle{empty}
\begin{sidewaystable*}[h]
	\centering
	\fontsize{8}{20}
	\selectfont
	\setlength\tabcolsep{1pt}
\begin{tabular*}{\columnwidth}{l@{\extracolsep{\fill}}lcccccccccccccccc}
	\toprule
                                    &            & \multicolumn{3}{c}{\textbf{agnews}}  & \multicolumn{3}{c}{\textbf{yahoo}}   & \multicolumn{3}{c}{\textbf{yelp-full}} & \multicolumn{3}{c}{\textbf{mnli}}    & \multicolumn{3}{c}{\textbf{qqp}}     & \multirow{2}{*}{\textbf{unsup}} \\
                                    &            & \textbf{0} & \textbf{1} & \textbf{2} & \textbf{0} & \textbf{1} & \textbf{2} & \textbf{0}  & \textbf{1}  & \textbf{2} & \textbf{0} & \textbf{1} & \textbf{2} & \textbf{0} & \textbf{1} & \textbf{2} &                                 \\
				    \hline
\multirow{3}{*}{\textbf{agnews}}    & \textbf{0} & 81.86      & 82.64      & 80.86      & 93.54      & 93.20      & 92.68      & 93.52       & 93.26       & 92.02      & 93.76      & 93.34      & 92.70      & 93.14      & 92.84      & 92.14      & 65.48                           \\
                                    & \textbf{1} & 81.88      & 81.92      & 79.80      & 93.16      & 93.90      & 91.40      & 93.06       & 93.52       & 90.66      & 93.16      & 93.50      & 92.00      & 92.70      & 93.56      & 93.36      & 74.72                           \\
                                    & \textbf{2} & 82.42      & 80.16      & 82.32      & 92.76      & 90.08      & 93.48      & 92.24       & 91.16       & 93.24      & 92.62      & 91.86      & 93.22      & 92.92      & 91.60      & 91.92      & 59.76                           \\
				    \hline
\multirow{3}{*}{\textbf{yahoo}}     & \textbf{0} & 75.06      & 75.14      & 74.60      & 54.70      & 41.98      & 53.58      & 75.20       & 75.06       & 74.46      & 75.22      & 75.16      & 74.98      & 74.88      & 74.28      & 73.78      & 42.22                           \\
                                    & \textbf{1} & 74.70      & 75.48      & 74.00      & 54.72      & 49.98      & 55.18      & 75.02       & 75.42       & 73.48      & 74.90      & 75.26      & 74.52      & 74.90      & 75.20      & 73.60      & 51.26                           \\
                                    & \textbf{2} & 73.92      & 71.64      & 75.44      & 44.68      & 27.12      & 50.92      & 73.86       & 70.72       & 74.92      & 74.62      & 73.82      & 75.04      & 74.24      & 69.50      & 75.40      & 36.02                           \\
				    \hline
\multirow{3}{*}{\textbf{yelp-full}} & \textbf{0} & 68.28      & 66.78      & 67.98      & 68.02      & 66.76      & 67.72      & 33.98       & 40.10       & 30.94      & 68.02      & 68.04      & 67.52      & 68.48      & 67.76      & 67.76      & 40.08                           \\
                                    & \textbf{1} & 65.98      & 67.44      & 66.36      & 66.24      & 67.34      & 64.90      & 31.48       & 36.80       & 30.24      & 64.84      & 68.22      & 66.74      & 65.18      & 68.00      & 65.64      & 40.34                           \\
                                    & \textbf{2} & 64.02      & 63.00      & 68.00      & 66.42      & 64.00      & 68.40      & 20.42       & 29.46       & 20.86      & 54.00      & 39.48      & 68.32      & 63.20      & 56.42      & 68.32      & 20.08                           \\
				    \hline
\multirow{3}{*}{\textbf{mnli}}      & \textbf{0} & 88.22      & 88.06      & 52.78      & 88.30      & 88.04      & 58.52      & 87.60       & 87.82       & 53.46      & 35.50      & 31.74      & 34.70      & 88.14      & 87.98      & 57.58      & 35.62                           \\
                                    & \textbf{1} & 85.00      & 88.38      & 48.34      & 86.64      & 88.46      & 52.98      & 86.12       & 87.88       & 49.50      & 35.14      & 34.36      & 35.34      & 86.48      & 87.86      & 54.80      & 36.22                           \\
                                    & \textbf{2} & 81.58      & 79.76      & 88.22      & 81.28      & 81.28      & 87.76      & 81.40       & 79.54       & 80.18      & 36.86      & 45.66      & 36.08      & 83.84      & 81.58      & 88.30      & 43.42                           \\
				    \hline
\multirow{3}{*}{\textbf{qqp}}       & \textbf{0} & 85.96      & 86.12      & 36.34      & 86.14      & 86.08      & 36.34      & 86.54       & 86.12       & 36.52      & 85.84      & 86.18      & 38.92      & 77.40      & 77.30      & 36.44      & 54.74                           \\
                                    & \textbf{1} & 85.24      & 86.18      & 36.35      & 85.92      & 86.16      & 36.35      & 85.46       & 86.84       & 36.38      & 85.52      & 86.30      & 36.40      & 72.16      & 76.38      & 36.38      & 36.68                           \\
                                    & \textbf{2} & 74.70      & 80.70      & 86.44      & 78.88      & 81.42      & 86.28      & 83.02       & 80.94       & 86.48      & 77.78      & 84.76      & 86.48      & 78.48      & 77.52      & 77.10      & 40.20 \\
				    \bottomrule
\end{tabular*}
	\caption{\textbf{Multi-task training full results}~---~reported accuracy results for each train and evaluation PVP combination. Different training tasks and PVPs are arranged in columns and evaluation tasks and PVPs in rows. The "All" column corresponds to training on all tasks together and "Unsup." column represents the unsupervised baseline without any fine-tuning.}\label{tab:multi-on-train-set}
\end{sidewaystable*}

\begin{table*}
	\centering
	\fontsize{8}{12}
	\selectfont
	\begin{tabular}{lllllllllllllll}
		\toprule
		\multicolumn{1}{l}{\textbf{}}       & \textbf{}  & \multicolumn{3}{c}{\textbf{wo/agnews}}                                                              & \multicolumn{3}{c}{\textbf{wo/yahoo}}                                                               & \multicolumn{3}{c}{\textbf{wo/yelp-full}}                                                           & \multicolumn{3}{c}{\textbf{wo/mnli}}                                                                & \multicolumn{1}{c}{\multirow{2}{*}{\textbf{wo/unsup}}} \\
		\multicolumn{1}{l}{\textbf{}}       & \textbf{}  & \multicolumn{1}{c}{\textbf{0}} & \multicolumn{1}{c}{\textbf{1}} & \multicolumn{1}{c}{\textbf{2}} & \multicolumn{1}{c}{\textbf{0}} & \multicolumn{1}{c}{\textbf{1}} & \multicolumn{1}{c}{\textbf{2}} & \multicolumn{1}{c}{\textbf{0}} & \multicolumn{1}{c}{\textbf{1}} & \multicolumn{1}{c}{\textbf{2}} & \multicolumn{1}{c}{\textbf{0}} & \multicolumn{1}{c}{\textbf{1}} & \multicolumn{1}{c}{\textbf{2}} & \multicolumn{1}{c}{}    \\
		\hline
		\multirow{3}{*}{\textbf{agnews}}    & \textbf{0} & 77.16                          & 79.32                          & 76.22                          & 91.94                          & 91.66                          & 91.86                          & 91.78                          & 91.80                          & 91.84                          & 91.62                          & 91.72                          & 91.68                          & 65.48                              \\
		& \textbf{1} & 80.70                          & 81.68                          & 79.74                          & 91.88                          & 92.00                          & 91.74                          & 92.00                          & 91.88                          & 91.82                          & 91.98                          & 91.76                          & 91.68                          & 74.72                              \\
		& \textbf{2} & 78.34                          & 80.14                          & 77.44                          & 91.96                          & 91.86                          & 91.84                          & 92.16                          & 92.00                          & 91.96                          & 91.94                          & 91.72                          & 91.82                          & 59.76                              \\
		\hline
		\multirow{3}{*}{\textbf{yahoo}}     & \textbf{0} & 72.60                          & 71.80                          & 71.92                          & 54.18                          & 46.40                          & 52.98                          & 72.22                          & 72.24                          & 72.02                          & 72.28                          & 71.88                          & 72.48                          & 42.22                              \\
		& \textbf{1} & 72.38                          & 71.94                          & 72.16                          & 56.38                          & 51.30                          & 54.94                          & 72.22                          & 71.98                          & 72.06                          & 72.42                          & 71.96                          & 72.26                          & 51.26                              \\
		& \textbf{2} & 70.86                          & 68.60                          & 72.28                          & 41.30                          & 31.88                          & 51.10                          & 70.70                          & 69.56                          & 72.00                          & 71.70                          & 71.44                          & 72.34                          & 36.02                              \\
		\hline
		\multirow{3}{*}{\textbf{yelp-full}} & \textbf{0} & 65.72                          & 65.00                          & 65.40                          & 65.72                          & 65.16                          & 65.50                          & 36.22                          & 37.30                          & 34.24                          & 65.82                          & 65.50                          & 65.54                          & 40.08                              \\
		& \textbf{1} & 64.50                          & 65.46                          & 64.42                          & 64.48                          & 65.84                          & 65.00                          & 36.90                          & 37.22                          & 35.34                          & 65.02                          & 65.40                          & 65.28                          & 40.34                              \\
		& \textbf{2} & 64.82                          & 64.18                          & 65.32                          & 65.18                          & 64.60                          & 65.36                          & 22.70                          & 31.20                          & 27.90                          & 61.34                          & 52.50                          & 65.40                          & 20.08                              \\
		\hline
		\multirow{3}{*}{\textbf{mnli}}      & \textbf{0} & 85.82                          & 85.54                          & 62.28                          & 85.84                          & 85.36                          & 66.48                          & 85.92                          & 85.56                          & 64.84                          & 35.14                          & 33.16                          & 35.28                          & 35.62                              \\
		& \textbf{1} & 84.76                          & 85.84                          & 58.04                          & 84.40                          & 85.94                          & 62.62                          & 84.04                          & 85.80                          & 61.02                          & 35.98                          & 35.76                          & 35.88                          & 36.22                              \\
		& \textbf{2} & 81.02                          & 79.78                          & 85.48                          & 81.76                          & 81.50                          & 85.72                          & 81.52                          & 80.80                          & 85.40                          & 41.42                          & 42.70                          & 41.08                          & 43.42                             \\
		\bottomrule
	\end{tabular}
	\caption{Multi-task using logits}\label{tab:multi-use-logits-full}
\end{table*}

%\includegraphics{single-on-train-set-horizontal copy}


\end{document}
